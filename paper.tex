\documentclass[9pt]{article}

\usepackage{verbatim}
\usepackage{amsmath}
\usepackage{enumerate}
\usepackage{url}
\usepackage[margin=.75in]{geometry}
\usepackage{amsthm}

\linespread{1}

%Title information for the entirety of the class document
\title{Hoverboard Construction and Experimentation using ROS}
\author{Kevin Chiang, Tony Schneider, Angela Yu, Baoling Zhao}
\date{9/26/2013}
\begin{document}
\maketitle

\section*{Introduction}
A robot can be thought of as a series of simple components working in unison to produce seemingly complicated behavior.  However, this requires each component to work as a cohesive whole -- a task which is daunting (to say the least). \emph {ROS} (the \textbf{R}obot \textbf{O}perating \textbf{S}ystem), an open source framework for robotics programming, was created to facilitate the use and communication of an arbitrary amount of these components (called \emph{nodes}). 

Using ROS in conjunction with several pre-written ROS nodes used to control radio and serial communication provides a platform for learning the basics of robotics. A hovercraft was constructed using relatively simple materials (i.e., styrofoam, tape, and plastic), and powered via 6 GW-EDF40 thrusters and a battery.  ROS was used to control each individual thruster with an Xbox controller, as well as fire thrusters in combination with one another to provide for more complicated translational movements.  This experience provided first hand experience in the construction and design of a simple robot, as well as an understanding of the fine tuning and manual evaluation required when working with mechanical systems.

\section*{Hovercraft Construction}
--Base\\
--Thruster Layout\\
--Skirt Design\\
--Weight management/arrangement\\

\section*{ROS Setup and Experiments}
--  Essentially section 5

\section*{Hovercraft Experiments}
-- Essentially section 6

\end{document}